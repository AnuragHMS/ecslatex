
\section{Edge Node Technologies}
\label{edge_node_technologies}
\subsection{What is Edge Computing?}

Cloud Computing offers powerful computation capabilites to almost anywhere across the globe given a fast and reliable connection. However, in a lot of remote locations, this access is not always readily available. Sometimes critical applications require immediate response by processing and analytics. If this local environment sends data up to the cloud and waits for real-time analytics to be performed, the remote location could struggle to get any or will receive a very delayed response. One such case could be at an oil rig in the middle of the ocean with limited connectivity to the internet. If the system is monitoring pressures within its pipes and if any excess pressure is seen, then the pipes may need to be bled out. However, to recognise this change in pressure over time previous data and all new data needs to be seen. If any immediate response is needed to change the pressure in the pipes, sending this data to the cloud would not have the required latency response as would be expected in such a crucial system to ensure system stability.

Besides, with the rapid development of Internet-of-Things (IoT), all kinds of electrical devices such as sensors and Internet-connected devices have become the data producers at the edge of network, generating more than billions of information within few years. Overwhelmed with enormous data generation simultaneously, data transportation has been a bottleneck for the traditional cloud computing, resulting not only latency due to the longer processing time but also overloading its network transmission bandwidth. Let say an autonomous vehicle that uses auto-pilot mode to drive itself to the predetermined destination using intelligence, requires real-time processing to make correct decisions. It produced about one Gigabyte of data per second. If these huge data to be sent to the cloud to do the processing, the response will take longer time.

The solution that arises from this is to have the processing and data analytics localised to the environment. However, how you ensure reliability, roll out updates, collect data securely and from many different devices simultaneously are the main difficulties in this solution rather than each sensor just being connected to the cloud directly and send data that way, which was just discussed could have disastrous latency-specific response issues. This solution is known as Edge Computing. “Edge” from Edge Computing is defined to be any devices with extended cloud services at the edge of network, in between the data sources and cloud data centres, covering the downstream of data on the cloud services and upstream data on the IoT devices. IoT devices that collect data (such as network or security measurements) would send their data to this local hardware running a server. This local server would contain processing, storage and networking capabilities that would allow connectivity to and within the entire remote location. One can reliably ensure that any control systems get the reliable and response they need in the correct time

To be specific, edge computing enables custom-designed edge native application that have the functionality of the extended cloud services such that computing the data, to be deployed into the user connected devices on the edge such as cell phones, cameras, vehicles, IoT devices in general, which can be referred as user equipment (UE) in the telecommunication industry. Edge native application usually takes up one or more attributes of Edge Computing which are bandwidth scalability, low-latency offload, privacy-preserving denaturing and Wide-Area-Network (WAN) failure resiliency. Comparing most of the studied edge native application, there are four main categories of the application:
\begin{itemize}
    \item \textbf{Single User Interactive}: This application involves a single user interacting through a user equipment (UE) with distributed application services. The interaction between the users is negligible though huge number of users using the services simultaneously. Example includes many augmented reality applications like virtual desktop infrastructure. Our research focuses on this application.
    \item \textbf{Multi-User Interactive}: This application is the advancement of single user interactive and includes many similar features with additional characteristic such as interaction between the users. Example includes multi-player video gaming. 
    \item \textbf{Edge Analytics}: Application where data collection and processing by distributed UEs were transferred to a centralized location to gain insight that drive operational action. However, it has implications of cost and latency in data transmission. Examples include intelligent processing of surveillance videos.
    \item \textbf{IoT sensors}: Application where data from the distributed sensor and UEs was collected for the analytic functions in order to provide control and assist. Examples are autonomous vehicles and distributed traffic monitoring services. 
\end{itemize}

\subsection{Instrumented Client Application}

AWS provides its own Edge Computing solution to their own Cloud services. Using IoT Greengrass, one can deploy Lambda functions, local resource access, device connectivity (e.g. to Sensors, Cameras, Network cards etc...). Greengrass allows functionality such that you can setup a group via the online dashboard. The online dashboard when the 'deploy' button is hit, every device with the Greengrass Core installed and with access to the Greengrass group via correct certifications will then update when a connection to the cloud can be established. Implementation of this is shown later in the implementation section.

Greengrass lets developers use their Cloud-Based Lambda functions and deploy these of the device allowing that the runtime environment is setup on the device. E.g. In order to run NodeJS Lambda Functions, one has to install the prerequisites and ensure Node version 12.x or similar is setup on the device. While this can be seen as tedious, it is in the developers hands to be able to automate this process by using scripts to setup the Greengrass core as well as any runtime libraries required.

Specialize in monitoring, analysis and control over the networks, the AdvantEDGE edge emulation platform able to build frameworks for modelling and measuring edge computing infrastructure as a tool for application characterization to provides strong basis of understanding and improvements on how the networks behaved in real world based on simulated environments. The modelling focuses on applying the network characteristics such as latency, jitter, packet loss and data throughput to measure the performance and quality of network system in delivering the service to the network users. The system design and implementation of the applications will be discussed further in the document.

%Research into what AdvantEdge is and how it can be used to emulate edge-nodes and network configurations to edge nodes