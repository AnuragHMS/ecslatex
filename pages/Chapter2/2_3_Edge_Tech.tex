
\section{Edge Node Technologies}
\label{edge_node_technologies}
\subsection{What is Edge Computing?}

Cloud Computing offers powerful computation capabilities to almost anywhere across the globe given a fast and reliable connection. However, in a lot of remote locations, this access is not always readily available. Sometimes, critical applications require immediate response by processing and analysis. If this local environment sends data up to the cloud and waits for real-time analysis to be performed, the remote location could struggle to get any or will receive a very delayed response. One such case could be at an oil rig in the middle of the ocean with limited connectivity to the internet. If the system is monitoring pressures within its pipes and if any excess pressure is seen, then the pipes may need to be bled out. However, to recognise this change in pressure over time previous data and all new data needs to be seen. If any immediate response is needed to change the pressure in the pipes, sending this data to the cloud would not have the required latency response as would be expected in such a crucial system to ensure system stability.

Besides, with the rapid development of Internet-of-Things (IoT), all kinds of electrical devices such as sensors and Internet-connected devices have become the data producers at the edge of network, generating more than billions of information within few years. Overwhelmed with enormous data generation simultaneously, data transportation has been a bottleneck for the traditional cloud computing, resulting not only latency due to the longer processing time but also overloading its network transmission bandwidth \cite{lei_roy_xiaoming_stefano_zbigniew_hannes_2013}. For example, an autonomous vehicle which drives itself to the predetermined destination, requires real-time processing to make correct decisions. It produces about one Gigabyte of data per second \cite{mark_2013}. If these huge data were to be sent to the cloud for processing, the response will take a longer time.

The solution to solve this problem is to have the processing and data analysis localised on the environment. However, ensuring reliability, rolled-out updates, secured data collection and from many different devices simultaneously are the main difficulties in this solution rather than having each sensor just being connected to the cloud directly and send data that way, which was just discussed could have disastrous latency-specific response issues. This solution is known as Edge Computing according to Shi et al. \cite{w_h_q_w_2017}, 

\begin{quote}
    \textit{"Edge computing is a new computing mode of network edge execution. The downlink data of edge computing represents cloud service, the uplink data represents the Internet of Everything, and the edge of edge computing refers to the arbitrary computing and network resources between the data source and the path of cloud computing center."}
\end{quote}

IoT devices that collect data (such as network or security measurements) would send their data to this local hardware running a server. This local server would contain processing, storage and networking capabilities that would allow connectivity to and within the entire remote location. One can reliably ensure that any control systems get the reliable and response they need in the correct time. 

\subsection{Edge Native Application}
Deployed to the edge of network with its extended cloud services, edge native application usually takes up one or more attributes of Edge Computing which are bandwidth scalability, low-latency offload, privacy-preserving denaturing and Wide-Area-Network (WAN) failure resiliency \cite{satyanarayanan_klas_silva_mangiante_2019}. Comparing most of the studied edge native application, there are four main categories of the application:  
\begin{itemize}
    \item \textbf{Single User Interactive}: An independent application with a single instance that execute through a user equipment (UE) in a single location. The simultaneous interaction between users over the same network/services is negligible. Our research focuses on this edge native application.
    \item \textbf{Multi-User Interactive}: The advancement of the single user interactive application with many similar features and additional characteristics where the edge application is formed by a group of edge applications. For example, multi-player video gaming. 
    \item \textbf{Edge Analytics}: Application where data collection and processing by distributed UEs were transferred to a centralized location to gain insight that drive operational action. However, it has implications of cost and latency in data transmission. Examples include intelligent processing of surveillance videos.
    \item \textbf{IoT sensors}: Application where data from the distributed sensor and UEs was collected for the analytic functions in order to provide control and assist. Examples are autonomous vehicles and distributed traffic monitoring services. 
\end{itemize}

\subsection{Instrumented Client Application}

AWS provides its own Edge Computing solution to their own Cloud services. Using IoT Greengrass, one can deploy Lambda functions, local resource access, device connectivity (e.g. to Sensors, Cameras, Network cards etc...). Greengrass allows functionality such that you can set up a group via the online dashboard. A deploy option is available in the online dashboard where  every device with the Greengrass Core installed and with access to the Greengrass group via correct certifications will update when a connection to the cloud can be established when the option is selected. Implementation of this is shown later in the implementation section.

Greengrass allows developers to use their Cloud-Based Lambda functions and deploy these of the device allowing that the runtime environment is setup on the device. E.g. In order to run NodeJS Lambda Functions, one has to install the prerequisites and ensure Node version 12.x or similar is setup on the device. While this can be seen as tedious, it is in the developers hands to be able to automate this process by using scripts to set up the Greengrass core as well as any runtime libraries required.

Specialize in monitoring, analysis and control over the networks, the AdvantEDGE edge emulation platform is able to build frameworks for modelling and measuring edge computing infrastructure as a tool for application characterization to provide strong basis of understanding and improvements on how the networks behaved in real world based on simulated environments. The modelling focuses on applying the network characteristics such as latency, jitter, packet loss and data throughput to measure the performance and quality of network system in delivering the service to the network users. The system design and implementation of the applications will be discussed further in the document.

%Research into what AdvantEdge is and how it can be used to emulate edge-nodes and network configurations to edge nodes